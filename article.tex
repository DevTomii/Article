\documentclass[10pt,twoside,slovak,a4paper]{article}

\usepackage[slovak]{babel}
\usepackage[IL2]{fontenc}
\usepackage[utf8]{inputenc}
\usepackage{graphicx}
\usepackage{url} 
\usepackage{hyperref}

\usepackage{cite}

\pagestyle{headings}

\title{Semestrálny projekt v predmete Métody inžinierskej práce, akademický rok 2022/2023, vedenie: Igor Stupavský}

\author{Tomáš Drga\\[2pt]
	{\small Slovenská technická univerzita v Bratislave}\\
	{\small Fakulta informatiky a informačných technológií}\\
	{\small \texttt{xdrga@stuba.sk}}
	}

\date{\small 10.10.2022}



\begin{document}

\maketitle

\begin{abstract}
\ldots
\end{abstract}

\section{Úvod}

Ako zvolenú tému som si vybral detekovanie podvodov v onlinech hrách. O túto tématiku som sa zaujímal už aj v minulosti, a tak mi prišla aj ako vhodná téma pre moju semestrálnu prácu v predmete Métody inžinierskej práce. V tomto článku sa dozviete ako sa doposiaľ detekovali podvody v online hrách, a taktiež ako sa to zmenilo s príchodom nových technológií.
Motivujte čitateľa a vysvetlite, o čom píšete. Úvod sa väčšinou nedelí na časti.

\section{Historia}

V tejto sekcií si povieme ako sa používal anticheat doteraz a prečo prišiel čas na zmenu. Anticheat fungoval pomerne jednoducho nakoľko nebolo možné overiť či hráč naozaj podvádza v dôsledku nedostatku informácií zo servera. Nedostatok informacií si môžeme spojiť so zabezpečením hry. No keďže nie je dostatok informácií na odhalenie podvodov museli herné spoločnsoti prísť s novými riešeniami nakoľko prichádzali o veľké množstvo peňazí nakoľko hry strácali na popularite.
